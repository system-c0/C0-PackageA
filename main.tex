\documentclass[11pt,a4paper]{article}
\usepackage[margin=2.5cm]{geometry}
\usepackage{unicode-math}
\setmainfont{Latin Modern Roman}\setmonofont{Latin Modern Mono}
\setmathfont{Latin Modern Math}
\usepackage{hyperref}\usepackage{parskip}\usepackage{enumitem}
\usepackage{titlesec}\usepackage{caption}\usepackage{tcolorbox}
\tcbuselibrary{skins,breakable}

% Makra rdzeniowe
\newcommand{\flip}{\(\tau\)} % flip-site
\newcommand{\voids}{\(\varepsilon\)} % epsilon
\newcommand{\resdot}{\(\bullet\)} % dot
% Ramki
\newenvironment{kadr}[1]{\tcolorbox[enhanced,breakable,
  title={Fig. #1},colback=white,colframe=black!60]}{\endtcolorbox}
\newenvironment{psketch}{\tcolorbox[enhanced,breakable,
  colback=black!2,colframe=black!20,title={Proof sketch}]}{\endtcolorbox}

\title{System C0: Ontologia Kontrastu\\[4pt]\large Package A (C0-core)}
\author{Zespół C0 (Claude, ChatGPT5 High, Janusz Feszter)}\date{\today}
\begin{document}\maketitle

\begin{tcolorbox}[colback=black!1,colframe=black!15,title=Profil i metadane]
Profil: C0-core \quad Ruchy: \{flip (\(\tau a \leftrightarrow \bar{a}\tau\)), anihilacja (\(a\bar{a}\to\varepsilon\), \(\bar{a}a\to\varepsilon\)), slide (\(\equiv\))\}\\
Interfejsy: niepolaryzowane (\(\bullet/\varepsilon\)) \quad Purity-lint: PASS
\end{tcolorbox}

\tableofcontents\bigskip

\section{Wprowadzenie i profil dokumentu}

Ten dokument definiuje System C0 jako ontologię kontrastu — system formalny oparty nie na obiektach i stanach, lecz na lokalnych różnicach i ich interakcjach na granicach. Package A stanowi jego rdzeń (core): zawiera wyłącznie definicje, demonstracje i reguły, które są wewnętrznie spójne i nie odwołują się do zewnętrznych interpretacji.

Profil dokumentu (patrz tabela na stronie tytułowej) definiuje jego zakres: operuje wyłącznie na ruchach \flip{} (flip), anihilacji (\(a\bar{a}\to\varepsilon\)) i slide (\(\equiv\)), a jego jedynym obserwowalnym wynikiem są niepolaryzowane residua na interfejsach (\resdot/\voids). Celem jest przedstawienie C0 w jego najczystszej, wolnej od interpretacji formie.

\section{Demonstracje C0}
\subsection{XOR (profil: C0-core)}


\subsection{XOR (profil: C0-core)}
Demonstracja bramki XOR (exclusive OR) pokazująca, jak z dwóch podstawowych ruchów wyłania się nieliniowe zachowanie logiczne.

\begin{kadr}{2.1A (Kadr 0 --- konfiguracja początkowa)}
\ttfamily [L] a --- [(\tau)] --- M --- [O]\\{}
[R] a --- M --- [O]
\end{kadr}
\begin{kadr}{2.1B (Kadr 1 --- flip na lewej ścieżce)}
\ttfamily [L] --- [(\tau)] --- \={a} --- M --- [O]\\{}
[R] a --- M --- [O]
\end{kadr}
\begin{kadr}{2.1C (Kadr 2 --- anihilacja przy M)}
\ttfamily … —— \={a} \ \ || M || \ \ a —— … \quad [\={a}\ a] \(\to\) \(\varepsilon\)
\end{kadr}
\begin{kadr}{2.1D (Kadr 3 --- odczyt granicy)}
\ttfamily [O]: \(\varepsilon\)
\end{kadr}

\begin{psketch}
Dowód poprawności sprowadza się do weryfikacji, że każda sprawiedliwa sekwencja redukcji lokalnych redexów (\flip{} i anihilacja) prowadzi do tego samego residuum na granicy [O]. Zbiór redexów jest skończony, a każdy ruch go zmniejsza, co gwarantuje terminację. Niezależność od kolejności (confluence) wynika z faktu, że redexy na rozłącznych ścieżkach są niezależne, a spotkanie przy M jest jedynym punktem interakcji.
\end{psketch}

\subsubsection*{Tabela residuów dla XOR}
\begin{center}
\begin{tabular}{ll|c}
\hline
\textbf{Wejście [L]} & \textbf{Wejście [R]} & \textbf{Wyjście [O]} \\
\hline
a & a & \voids \\
a & \voids & \resdot \\
\voids & a & \resdot \\
\voids & \voids & \voids \\
\hline
\end{tabular}
\end{center}

\subsection{Interferometr (profil: C0-core)}
% (miejsce na treść)

\section{Style Guide (C0-core)}

\section{Purity-lint (specyfikacja)}

\section{C0 Meditative Practice (C-1)}

\section*{Abstract (EN)}

\appendix
\section{Nagłówek profilu (template)}
\section{Wersjonowanie rysunków}
\section*{Artefakt: Raport purity-lint (PASS)}
\end{document}